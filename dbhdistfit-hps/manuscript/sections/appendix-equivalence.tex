% !TEX root = ../main.tex
\section{Proof of Equivalence Between Estimators}
\label{sec:appendix-equivalence}

This appendix presents a rigorous proof that the weighted estimator introduced
in Section~\ref{sec:methods} is equivalent to the size-biased estimator of
\citet{van1986, ducey2015}. Throughout, we assume a fixed binning scheme for
DBH classes $\mathcal{I}$ with midpoints $x_i$ and employ the notation defined
in the main text.

\subsection{Preliminaries}

Let $X$ denote diameter at breast height (DBH) with underlying PDF
$f(x;\boldsymbol{\theta})$ and CDF $F(x;\boldsymbol{\theta})$ parameterised by
$\boldsymbol{\theta}\in\Theta$. Under horizontal point sampling (HPS) with basal
area factor $C_{BA}$, the inclusion probability of a stem of DBH $x$ is
proportional to its basal area, i.e., $\pi(x)\propto x^2$. The size-biased PDF
of order two is therefore
\begin{equation}
  f_{\mathrm{sb}}(x;\boldsymbol{\theta})
  =
  \frac{x^{2} f(x;\boldsymbol{\theta})}{\kappa(\boldsymbol{\theta})},
  \qquad
  \kappa(\boldsymbol{\theta})
  =
  \int_{0}^{\infty} x^{2} f(x;\boldsymbol{\theta})\,\mathrm{d}x,
\end{equation}
which is well-defined whenever $\kappa(\boldsymbol{\theta}) < \infty$.

Let $t_i$ denote the observed HPS tally (aggregated across plots) for bin $i$,
and let $\hat{y}_i$ denote the corresponding expanded stand table value defined
in Equation~\eqref{eq:stand-table}. Define the HPS expansion factor
\begin{equation}
  f_E(x; C_{BA}) = \frac{40000 \, C_{BA}}{\pi x^2},
\end{equation}
and the compression factor $f_C(x; C_{BA}) = f_E(x; C_{BA})^{-1}$.

\subsection{Reference estimator}

The size-biased estimator minimises the sum of squares between observed tallies
and the size-biased PDF evaluated at bin midpoints:
\begin{equation}
  Z_{\mathrm{C}}(\boldsymbol{\theta}, s)
  =
  \sum_{i \in \mathcal{I}}
  \Big[t_i - s\, f_{\mathrm{sb}}(x_i; \boldsymbol{\theta})\Big]^2,
  \label{eq:appendix-control}
\end{equation}
where $s$ is a scaling parameter converting continuous densities to discrete
bin counts. The minimiser $(\hat{\boldsymbol{\theta}}_{\mathrm{C}},\hat{s}_{\mathrm{C}})$ is
obtained by solving the normal equations associated with
\eqref{eq:appendix-control} using standard non-linear least squares algorithms.

\subsection{Weighted estimator}

The weighted estimator operates on expanded stand table data with weights
matching the inverse of the expansion factor:
\begin{equation}
  Z_{\mathrm{T}}(\boldsymbol{\theta}, s)
  =
  \sum_{i \in \mathcal{I}}
  w_i^2 
  \Big[\hat{y}_i - s\, f(x_i; \boldsymbol{\theta})\Big]^2,
  \qquad
  w_i = f_C(x_i; C_{BA}).
  \label{eq:appendix-test}
\end{equation}
Recall $\hat{y}_i = f_E(x_i; C_{BA}) t_i$ by construction.

\subsection{Key lemma}

\begin{lemma}
For all $\boldsymbol{\theta}\in\Theta$ and $s>0$, the objectives
$Z_{\mathrm{C}}(\boldsymbol{\theta}, s)$ and $Z_{\mathrm{T}}(\boldsymbol{\theta}, s')$ differ
by a positive multiplicative constant after an appropriate reparameterisation of
$s$.
\end{lemma}

\begin{proof}
Insert $\hat{y}_i = f_E(x_i; C_{BA}) t_i$ and $w_i = f_C(x_i; C_{BA})$ into
\eqref{eq:appendix-test}:
\begin{align}
  Z_{\mathrm{T}}(\boldsymbol{\theta}, s)
    &= \sum_{i \in \mathcal{I}} \left[f_C(x_i; C_{BA}) f_E(x_i; C_{BA}) t_i
      - f_C(x_i; C_{BA}) s f(x_i; \boldsymbol{\theta})\right]^2 \\
    &= \sum_{i \in \mathcal{I}} \left[t_i - s \, f_C(x_i; C_{BA}) f(x_i; \boldsymbol{\theta})\right]^2.
\end{align}
Using the explicit form of $f_C$, we have
\begin{equation}
  f_C(x; C_{BA}) f(x; \boldsymbol{\theta})
  =
  \frac{\pi}{40000 C_{BA}} x^{2} f(x;\boldsymbol{\theta})
  =
  \frac{\kappa(\boldsymbol{\theta})}{40000 C_{BA}} f_{\mathrm{sb}}(x; \boldsymbol{\theta}).
\end{equation}
Let $s' = s\, \kappa(\boldsymbol{\theta})/(40000 C_{BA})$. Then
\begin{equation}
  Z_{\mathrm{T}}(\boldsymbol{\theta}, s)
  =
  \sum_{i \in \mathcal{I}} \left[t_i - s' f_{\mathrm{sb}}(x_i; \boldsymbol{\theta})\right]^2
  =
  Z_{\mathrm{C}}(\boldsymbol{\theta}, s').
\end{equation}
Hence $Z_{\mathrm{T}}$ and $Z_{\mathrm{C}}$ share identical level sets up to a
bijective scaling of $s$.
\end{proof}

\subsection{Equivalence of minimisers}

Let $(\hat{\boldsymbol{\theta}}_{\mathrm{C}}, \hat{s}_{\mathrm{C}})$ minimise
\eqref{eq:appendix-control}. By the lemma, there exists a corresponding scaling
$\hat{s}_{\mathrm{T}} = \hat{s}_{\mathrm{C}}\, 40000 C_{BA}/\kappa(\hat{\boldsymbol{\theta}}_{\mathrm{C}})$
such that $(\hat{\boldsymbol{\theta}}_{\mathrm{C}}, \hat{s}_{\mathrm{T}})$ minimises
\eqref{eq:appendix-test}. Conversely, any minimiser of
\eqref{eq:appendix-test} maps back to a minimiser of \eqref{eq:appendix-control}
by the inverse scaling. Therefore the estimators produce identical parameter
estimates $\hat{\boldsymbol{\theta}}_{\mathrm{C}} = \hat{\boldsymbol{\theta}}_{\mathrm{T}}$ and
the fitted PDFs coincide after transforming between tally and stand table space.

\subsection{Extensions}

The argument extends directly to unbinned data by replacing discrete sums with
integrals. Moreover, any objective function proportional to the squared error
(e.g., weighted least squares with constant variance within space) inherits the
same equivalence. Likelihood-based estimators such as Poisson regression also
admit analogous proofs: incorporating Horvitz--Thompson weights in the log-
likelihood yields the same score equations as using size-biased PDFs.

\subsection{Implications}

Because the minimisers coincide, diagnostic quantities (RSS, chi-square
statistics, fitted curves) are identical up to deterministic re-scaling between
tally and stand table space. In practice, the weighted formulation provides a
computationally convenient alternative that makes use of standard PDF
implementations and alleviates the need for bespoke size-biased forms.
