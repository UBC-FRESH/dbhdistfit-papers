% !TEX root = ../main.tex
\section{Introduction}

Horizontal point sampling (HPS), also referred to as angle count or Bitterlich
sampling \citep{bitterlich1947}, is widely used to quantify stand structure in
managed forests. Diameter distributions derived from HPS tallies underpin yield
modelling, inventory projection, and ecological assessments. A long-standing
challenge is that HPS tallies are intrinsically \emph{size-biased}: the
inclusion probability of a stem is proportional to its basal area, which varies
with the square of diameter. When standard probability density functions (PDFs)
are fit directly to expanded HPS stand tables, small diameter classes exert
excessive influence and the fitted distribution is biased \citep{van1986,
gove2000, ducey2015}. The conventional remedy is to derive size-biased forms of
candidate PDFs and fit them to the unexpanded tallies.

Deriving size-biased PDFs requires specialised algebra and bespoke software
implementations. Even when available, the resulting optimisation problems
often impose complex parameter bounds that reduce numerical stability. As a
result, practitioners occasionally bypass appropriate size-bias corrections,
leading to over-fitted distributions and poor downstream predictions.

This paper revisits the simplified method originally introduced in
\citet{paradis2019submitted}, extends it with a formal equivalence argument,
and packages the approach in a modern reproducible research workflow. The core
idea is to fit standard-form PDFs to expanded HPS stand tables while embedding
size-bias weights directly in the fitting algorithm. We show that the weighted
objective function is proportional to the objective of the reference
size-biased estimator, yielding identical parameter estimates and fitted
curves. The new workflow includes:
\begin{enumerate}
  \item a clean mathematical exposition establishing the equivalence between
  weighted and size-biased fitting;
  \item an updated computational experiment using permanent sample plots from
  Quebec to benchmark Weibull and Gamma distributions across multiple
  meta-plots; and
  \item a fully reproducible project scaffold containing LaTeX sources,
  notebooks, and Python scripts that regenerate all figures and tables.
\end{enumerate}

The remainder of the manuscript introduces notation and the reference method,
derives the weighted estimator, reports numerical comparisons, and discusses
implications for inventory analysis pipelines.
