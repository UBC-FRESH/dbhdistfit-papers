% !TEX root = ../main.tex
\section{Discussion}

The weighted estimator offers several advantages for practitioners and
researchers. First, it removes the requirement to derive and implement
size-biased PDFs for each candidate distribution. Instead, analysts can rely on
standard forms that are broadly available in statistical libraries. Second, the
weighted formulation relaxes parameter constraints. Many size-biased PDFs
impose non-linear bounds on shape parameters to ensure integrability; such
bounds complicate optimisation and inflate uncertainty estimates near the
boundary. By working with standard PDFs, we avoid these issues while preserving
statistical correctness.

Our proof of equivalence is based on non-linear least squares, but similar
results follow for likelihood-based estimators. In a Poisson framework, the
Horvitz--Thompson weights embedded in the log-likelihood produce the same score
equations as the size-biased formulation. Future work could extend the proof to
Bayesian models where size-bias corrections appear in the likelihood and priors
operate on standard parameterisations.

The reproducible workflow addresses modern expectations for transparency. The
project scaffold isolates data preparation, fitting, figure generation, and
manuscript compilation. Researchers can integrate new datasets by producing the
expected binned meta-plot file, while retaining the rest of the pipeline. The
Makefile ensures that figures and tables remain in sync with the manuscript,
reducing the risk of stale artefacts.

Limitations arise mainly from data availability. Our PSP-derived pseudo-HPS
dataset assumes that the variance structure of expanded tables mimics that of
true HPS tallies. Empirical validation using genuine HPS inventories would
strengthen the argument. Additionally, the current experiment focuses on
two-parameter Weibull and Gamma distributions; extending the comparison to
three-parameter families (e.g., generalised gamma) or mixture models is
straightforward with the proposed tooling.
