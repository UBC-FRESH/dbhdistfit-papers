\documentclass{article}

\title{Management and Policy Implications}

\author{}
\date{}

\begin{document}
\maketitle
\thispagestyle{empty}


Stem diameter distributions, commonly known to foresters as \emph{stand tables}, are one of the main sources of information used to develop stand-level silviculture prescriptions. Stand tables describe frequency of stems (per unit area---for example, stems per acre) by species and diameter class. Statistical models can be fitted to empirical stem diameter distributions (i.e. to binned forest inventory data). These models can then be used to approximate stand table information for stands that have not been inventoried, or for aggregated (stratified) stand data. 

The work presented here constitutes a key step in a larger modelling effort, which aims to link long-term wood supply models and short-term industrial fiber consumption models. We use the statistical models presented here to disaggregate output from long-term wood supply models (i.e. timber volume) for use as input to short-term industrial fiber consumption models (i.e. diameter-wise assortments of logs). This larger modelling effort aims to develop a methodology to estimate value-creation potential of the wood supply for the entire province of Quebec, in Canada. No suitable stem diameter distribution models have been published for principal commercial species in this area, so we developped a methodology---based on readily-available permanent sample plot data---for deriving stem diameter distribution models for 30 combinations of species and cover type found in Quebec. 

Although we found many examples of stem diameter distribution models in the literature, there is almost no information in the literature clearly describing a methodology for the relatively common forestry problem of fitting statistical models (e.g. Weibull, gamma, or exponential distributions) to forest inventory data with bounded domain (i.e. censored data, with all inventoried stem data contained in a bounded diameter interval). Specifically, this problem poses two non-trivial technical challenges, related to biased fits (induced by the bounded data domain) and quality of parameter error estimation (induced by our solution to the aforementioned bias problem). We describe a generic two-stage parameter-fitting methodology that resolves these two technical challenges, and present best-fit diameter distribution models for 30 combinations of species and cover type found in Quebec.  

\end{document}