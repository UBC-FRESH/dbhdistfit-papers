The generalized beta family of statistical distributions is useful for
modelling stem density distributions. % add citation
All members of the family can be derived from either the generalized beta distribution of the first kind (GB1) or the generalized beta distribution of the second kind (GB2).
The probability density functions (PDF) of $f_{GB1}$ and $f_{GB2}$ have the following forms (adapted from \citealp{ducey2015sizebiased})

\begin{equation}
f_{\text{GB1}}(x; \, b, p, q) = \frac{|a|x^{ap-1}\left[1 - (x/b)^a\right]^{q-1}}{b^{ap}\text{B}(p, q)}, \qquad 0 < y^a < b^a, b > 0, p > 0, q > 0
\end{equation}

and 

\begin{equation}
f_{\text{GB2}} (x; a, b, p, q) = \frac{|a|x^{ap-1}x^{q-1}}{b^{ap}\text{B}(p, q)\left[1 - (x/b)^a\right]^{p+q}}, \qquad a > 0, b > 0, p > 0, q > 0
\end{equation}

defined for $x > 0$, where $\text{B}(p, q)$ represents the beta function (not to be confounded with the beta or generalized beta distributions), which is given by

\begin{equation}
\text{B}(p, q) = \int_0^1 t^{p-1} (1 - t)^{q-1} dt.
\end{equation}

The generalized gamma GG distribution is a special case of both GB1 and GB2 distributions. The GG PDF has the following form

\begin{equation}
f_{\text{GG}}(x; a, \beta, p) = \frac{ax^{ap-1}e^{-\left(\frac{x}{\beta}\right)^a}}{\beta^{ap}\Gamma(p)}, \qquad a > 0, \beta > 0, q > 0
\end{equation}

defined for $x > 0$, where $\Gamma(p)$ represents the gamma function (not to be confounded with the gamma, or generalized gamma, distributions), which is given by

\begin{equation}
\Gamma(p) = \int_0^\infty x^{p-1}e^{-x} dx.
\end{equation}

The size-biased forms of GB1, GB2 and GG PDFs are given by (adapted from \citealp{ducey2015sizebiased})

\begin{align}
f^{\text{SB}}_{\text{GB1}}(x; a, b, p, q, \alpha) &= f_{\text{GB1}}(x; a, b, p + \alpha/a, q)\\
f^{\text{SB}}_{\text{GB2}}(x; a, b, p, q, \alpha) &= f_{\text{GB2}}(x; a, b, p + \alpha/a, q - \alpha/a)\\
f^{\text{SB}}_{\text{GG}}(x; a, \beta, p, \alpha) &= f_{\text{GG}}(x; a, \beta, p + \alpha/a)
\end{align}

where $\alpha$ corresponds to the \emph{order} of the distribution.
For the case of HPS, which is an area-based sampling technique, we need a size-biased distribution of order 2 (basal area being related to the square of diameter). 
Thus, we set $\alpha=2$ for the control scenario (i.e., this parameter
is fixed, and is not fit by the NLLS algorithm).

We can define standard and size-biased forms of the PDFs of several other distributions in terms of GB1, GB2, and GG PDFs.

The standard forms are given by
{\small

  \begin{align}
  f_{\text{IB1}}(x; b, p, q) &= f_{\text{GB1}}(x; -1, b, p, q) \\
  f_{\text{UG}}(x; b, d, q) &= \lim_{a \to \infty} f_{\text{GB1}}(x; a, b, d/a, q) \\
f_{\text{B1}}(x; b, p, q) &= f_{\text{GB1}}(x; 1, b, p, q) \\
f_{\text{B2}}(x; b, p, q) &= f_{\text{GB2}}(x; 1, b, p, q) \\
f_{\text{SM}}(x; a, b, q) &= f_{\text{GB2}}(x; a, b, 1, q) \\
f_{\text{Dagum}}(x; a, b, p) &= f_{\text{GB2}}(x; a, b, p, 1) \\
f_{\text{Pareto}}(x; b, p) &= f_{\text{GB1}}(x; -1, b, p, 1) \\
f_{\text{P}}(x; b, p) &= f_{\text{GB1}}(x; 1, b, p, 1) \\
f_{\text{GA}}(x; \beta, p) &= f_{\text{GG}}(x; 1, \beta, p) \\
f_{\text{W}}(x; a, \beta) &= f_{\text{GG}}(x; a, \beta, 1) \\
f_{\text{F}}(x; u, v) &= f_{\text{GB2}}(x; 1, v/u, u/2, v/2) \\
f_{\text{L}}(x; b, q) &= f_{\text{GB2}}(x; 1, b, 1, q) \\
f_{\text{IL}}(x; b, p) &= f_{\text{GB2}}(x; 1, b, p, 1) \\
f_{\text{Fisk}}(x; a, b) &= f_{\text{GB2}}(x; a, b, 1, 1) \\
f_{\text{U}}(x; b) &= f_{\text{GB1}}(x; 1, b, 1, 1) \\
f_{\tfrac{1}{2}\text{N}}(x; 0, \sigma) &= f_{\text{GG}}(x; 2, \sigma^2, 1/2) \\
f_{\chi^2}(x; p) &= f_{GG}(x; 1, 2, p) \\
f_{\text{EXP}}(x; \beta) &= f_{\text{GG}}(x; 1, \beta, 1) \\
f_{\text{R}}(x; \beta) &= f_{\text{GG}}(x; 2, \beta, 1) \\
f_{\text{LL}}(x; b) &= f_{\text{GB2}}(x; 1, b, 1, 1)
\end{align}
}%

The size-biased forms are given by (adapted from \citealp{ducey2015sizebiased})

{\small
\begin{align}
f^{\text{SB}}_{\text{IB1}}(x; b, p, q, \alpha) &= f_{\text{GB1}}(x; -1, b, p - \alpha, q),&  \alpha < p \\
f^{\text{SB}}_{\text{UG}}(x; b, d, q, \alpha) &= \lim_{a \to \infty} f_{\text{GB1}}(x; a, b, (d + \alpha)/a, q),&  \alpha > -d \\
f^{\text{SB}}_{\text{B1}}(x; b, p, q, \alpha) &= f_{\text{GB1}}(x; 1, b, p + \alpha, q),&  \alpha > -p \\
f^{\text{SB}}_{\text{B2}}(x; b, p, q, \alpha) &= f_\text{{GB2}}(x; 1, b, p + \alpha, q- \alpha),&  -p < \alpha < q \\
f^{\text{SB}}_{\text{SM}}(x; a, b, q, \alpha) &= f_{\text{GB2}}(x; a, b, 1 + \alpha/a, q - \alpha/a),&  -a < \alpha < aq \\
f^{\text{SB}}_{\text{Dagum}}(x; a, b, p, \alpha) &= f_{\text{GB2}}(x; a, b, p + \alpha/a, 1),&  -ap < \alpha < a \\
f^{\text{SB}}_{\text{Pareto}}(x; b, p, \alpha) &= f_{\text{GB1}}(x; -1, b, p - \alpha, 1),&  \alpha < p \\
f^{\text{SB}}_{\text{P}}(x; b, p, \alpha) &= f_{\text{GB1}}(x; 1, b, p + \alpha, 1),&  \alpha > -p \\
f^{\text{SB}}_{\text{GA}}(x; \beta, p, \alpha) &= f_{\text{GG}}(x; 1, \beta, p + \alpha),&  \alpha > -p \\
f^{\text{SB}}_{\text{W}}(x; a, \beta, \alpha) &= f_{\text{GG}}(x; a, \beta, 1 + \alpha/a),&  \alpha > -a \\
f^{\text{SB}}_{\text{F}}(x; u, v, \alpha) &= f_{\text{GB2}}(x; 1, v/u, u/2 + \alpha, v/2 - \alpha),&  -u/2 < \alpha < v/2 \\
f^{\text{SB}}_{\text{L}}(x; b, q, \alpha) &= f_{\text{GB2}}(x; 1, b, 1 + \alpha, q - \alpha),&  -1 < \alpha < q \\
f^{\text{SB}}_{\text{IL}}(x; b, p, \alpha) &= f_{\text{GB2}}(x; 1, b, p + \alpha, 1 - \alpha),&  -p < \alpha < 1 \\
f^{\text{SB}}_{\text{Fisk}}(x; a, b, \alpha) &= f_{\text{GB2}}(x; a, b, 1 + \alpha/a, 1 - \alpha/a),&  -a < \alpha < a \\
f^{\text{SB}}_{\text{U}}(x; b, \alpha) &= f_{\text{GB1}}(x; 1, b, 1 + \alpha, 1),&  \alpha > -1 \\
f^{\text{SB}}_{\tfrac{1}{2}\text{N}}(x; 0, \sigma, \alpha) &= f_{\text{GG}}(x; 2, \sigma^2, 1/2 + \alpha),&  \alpha > -1/2 \\
f^{\text{SB}}_{\chi^2}(x; p, \alpha) &= f_{\text{GG}}(x; 1, 2, p + \alpha),&  \alpha > -p \\
f^{\text{SB}}_{\text{EXP}}(x; \beta, \alpha) &= f_{\text{GG}}(x; 1, \beta, 1 + \alpha),&  \alpha > -1 \\
f^{\text{SB}}_{\text{R}}(x; \beta, \alpha) &= f_{\text{GG}}(x; 2, \beta, 1 + \alpha),&  \alpha > -1 \\
f^{\text{SB}}_{\text{LL}}(x; b, \alpha) &= f_{\text{GB2}}(x; 1, b, 1 + \alpha, 1 - \alpha),&  -1 < \alpha < 1
\end{align}
}%

We replicated our experiment on each of the 30 sub-datasets, with each of the N distributions defined above, for a total of $30N$ replicates.
